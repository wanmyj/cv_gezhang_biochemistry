% --- LaTeX CV Template - S. Venkatraman ---

% --- Set document class and font size ---

\documentclass[letterpaper, 11pt]{article}

% --- Package imports ---

\usepackage{hyperref, enumitem, longtable, amsmath, array}

% --- Page layout settings ---

% Set page margins
\usepackage[left=0.7in, right=0.8in, bottom=.8in, top=0.8in, headsep=0in, footskip=.2in]{geometry}

% Set line spacing
\renewcommand{\baselinestretch}{1.2}

% --- Page formatting settings ---

% Set link colors
\usepackage[dvipsnames]{xcolor}
\hypersetup{colorlinks=true, linkcolor=MidnightBlue, urlcolor=MidnightBlue}

% Set font to Libertine, including math support
\usepackage{libertine}
\usepackage[libertine]{newtxmath}

% Remove page numbering
\pagenumbering{gobble}

% Define font size and color for section headings
\newcommand{\headingfont}{\Large\color{OliveGreen}}

% --- CV section settings ---

% Note: each section of this table (Education, Awards, Publications etc.) is 
% stored in a two-column table. The left-hand column is narrow (1 inch) and is 
% meant to store dates. The right-hand column is wide (5.2 inches) and stores 
% the main text.  Sections in which each entry might have multiple lines 
% (e.g., Education) are stored in a 'SectionTable' environment). Sections in 
% which each entry might just have one line are stored in a 'SectionTableSingleSpace'
% environment. The only difference between the two environments is the line 
% spacing between each entry. Both environments take one argument, which is the
% title of the section. See main document for how these environments are used.

% Define settings for left-hand column in which dates are printed
\newcolumntype{R}{>{\raggedleft}p{1in}}

% Define 'SectionTable' environment
\newenvironment{SectionTable}[1]{
	\renewcommand*{\arraystretch}{1.7}
	\setlength{\tabcolsep}{10pt}
	\begin{longtable}{Rp{5.2in}} & #1 \\}
{\end{longtable}\vspace{-.3cm}}

% Define 'SectionTableSingleSpace' environment
\newenvironment{SectionTableSingleSpace}[1]{
	\renewcommand*{\arraystretch}{1.2}
	\setlength{\tabcolsep}{10pt}
	\begin{longtable}{Rp{5.2in}} & #1 \\[0.6em]}
{\end{longtable}\vspace{-.3cm}}

% --- Document starts here ---

\begin{document}

% --- Name and contact information ---

\begin{SectionTable}{\Huge Ge Zhang} & 
ge__zhang@163.com   $\;\boldsymbol{\cdot}\;$ 
www.yourwebsite.com $\;\boldsymbol{\cdot}\;$ 
19916545495 \newline
Citizenship: China
\end{SectionTable}

% --- Section: Research interests ---

\begin{SectionTable}{\headingfont Research interests}
& research on metabolic diseases including atherosclerosis, obesity, and diabetes.
\end{SectionTable}
% --- Section: Research interests2 ---

% --- Section: Education ---

\begin{SectionTable}{\headingfont Education}
2018 -- 2021 & 
\textbf{Fudan University} -- Shanghai, China \newline
MA in Biochemistry and Molecular Biology \newline 
Mentors: Si Zhang. \textit{GPA: X.YZ}. \\

2014 -- 2018 & 
\textbf{Nantong University} -- Nantong, China \newline
BA in Biotechnology - Animals \newline 
Mentors: Sheyu Lin. \textit{GPA: X.YZ}. \\

% --- Un-comment the next few lines if you want to include some courses you've taken ---

%& \textbf{Selected coursework}
%\begin{itemize}[itemsep=0pt, leftmargin=*]
%\item \textit{Statistics}: Asymptotic statistics, Mathematical statistics, Functional data analysis, High-dimensional statistics, Information theory
%\item \textit{Mathematics}: Measure theory, Functional analysis, Measure-theoretic probability with martingales
%\end{itemize}

\end{SectionTable}

% --- Section: Awards, scholarships, etc ---

\begin{SectionTableSingleSpace}{\headingfont Honors and scholarships}
2020 & 
Fudan University, Merit scholarshi \\

2020 &
Fudan University, Excellent teaching assistant\href{https://en.wikibooks.org/wiki/LaTeX/Hyperlinks}{link if you want}) \\

2017 &
Nantong University, The university's first-class scholarship and excellent learning model \\

2016 &
Nantong University, The university's second-class scholarship  \\

2015 &
Nantong University,The university's first-class scholarship 
\end{SectionTableSingleSpace}

% --- Section: Publications ---

\begin{SectionTable}{\headingfont Publications} 
2021 & 
\textbf{Activation of Platelet NLRP3 Inflammasome in Crohn’s Disease} \newline
Ge Zhang†, He Chen†, Yifan Guo†, Wei Zhang, Qiuyu Jiang, Si Zhang, Liping Han, She Chen and Ruyi Xue*. \newline
\textit{Frontiers in Pharmacology} \\

2021 & 
\textbf{The role of Sphingomyelin synthase 2 (SMS2) in platelet activation and its clinical significance} \newline
Y. Guo†, L Chang†, G. Zhang†, Z. Gao, H. Lin, Y. Zhang, L. Hu, S. Chen, B. Fan, S. Zhang, R. Xue. \newline
\textit{Thrombosis Journal} \\

2020 & 
\textbf{Methylation and serum response factor mediated in the regulation of gene ARRDC3 in breast cancer} \newline
S. Lin*, G. Zhang*, Y. Zhao, D. Shi, Q. Ye, Y. Li, S. Wang. \newline
\textit{Am J Transl Res}

\end{SectionTable}

% --- Section: Research experience ---

\begin{SectionTable}{\headingfont Research experience}
2018 -- 2021 &
\textbf{The role of X in thrombopoiesis and platelet granule secretion} \newline
Mentors: Si Zhang (Fudan University). \newline
1.The mechanism of platelet count increase caused by the loss of X: We found that the loss of X resulted in increased proliferation and higher maturity of megakaryocytes, thus generating more platelets through flow cytometry, transcriptome sequencing, WB and other methods.
2. The loss of X leads to reduced release of platelet particles : We found by flow cytometry and ELISA that the absence of X resulted in significantly reduced platelet particle release
3.We found that X interacts with follitin by IP and mass spectrometry, which was further verified by IP and immunofluorescence. By constructing truncated bodies, and by protein structure simulation, we identified the domains in which they interact.
4.Explore the role of X in chemotherapy: We explored the relationship between the degree of platelet activation and x expression in clinical chemotherapy patients. In addition, we further injected chemotherapy drugs into mice to explore the correlation between megakaryocyte proliferation, platelet count, activation and chemotherapy drugs in mice. \\

2018 -- 2021 &
\textbf{The role of Sphingomyelin synthase 2 (SMS2) in platelet activation and its clinical significance} \newline
Mentors:  Si Zhang (Fudan University). \newline
1. Weakening of thrombus formation in vivo due to SMS2 Deficiency:our results clearly indicate that SMS2 plays an
important role in thrombosis and hemostasis through FeCl3-induced thrombus formation in mouse mesenteric and tail bleeding.
2. I performed notify obligation of informed consent and collected blood samples. \\

2020 -- 2021 &
\textbf{Activation of Platelet NLRP3 Inflammasome in Crohn’s Disease} \newline
Mentors: Si Zhang (Fudan University). \newline
1.The role of NLRP3 inflammasome in Crohn's patients was determined by analyzing the blood routine data of healthy people and Crohn's patients, and by detecting the expression and activation of NLRP3 inflammasome in platelets and the production of ROS in platelets.
2.I completed the manuscript. \\

2015 -- 2018 &
\textbf{Methylation and serum response factor mediated in the regulation of gene ARRDC3 in breast cancer} \newline
Mentors: sheyu Lin (Nantong University). \newline
I completed Cell culture and viability, western blot, animal experiments. \\
\end{SectionTable}

% --- Section: Teaching experience ---

\begin{SectionTable}{\headingfont Teaching experience}
Fall 2019 & 
\textbf{Teaching assistant, Biochemical experiment course (Fudan University)} \newline
Prepare lessons and conduct pre-experiments together with accredited teachers; Guide students' experiment operation during experiment class; Grade students' lab reports. \newline
\textit{Average student rating: X/5.} \\

Summer  2020 & 
\textbf{Teaching assistant,Experimental course in molecular genetics (Fudan University)} \newline
Prepare lessons and conduct pre-experiments together with accredited teachers; Guide students' experiment operation during experiment class; Gradestudents' lab reports. \newline
\textit{Average student rating: X/5.}
\end{SectionTable}

% --- Section: work experience ---

\begin{SectionTable}{\headingfont Industry experience}
Summer 2020 &
\textbf{Name of company (Title of job or internship)} -- City, State \newline
Description of your responsibilities. Integer pretium semper justo. Proin risus. Nullam id quam. Nam neque. Phasellus at purus et lib ero lacinia dictum.  \\

202209-present &
\textbf{Abcam (Shanghai) Trading Co., LTD (scientific support)} -- Shanghai, China \newline
1.Answer customers' phone calls, answer customers' inquiries on product information, recommend suitable products to customers, solve the problems encountered in the experiment, and provide suggestions on experiment optimization;
2. Deal with customer complaint email, provide solutions for customer experiment, communicate with customer, solve customer problems;
3.Master basic molecular biology, cell biology and other experimental technology, to provide experimental solutions for customers.
4.Good communication and collaboration with team members to better solve customer problems. \\

202107-202207 &
\textbf{Fudan University  (Research assistant)} -- Shanghai, China \newline
research focusing on the pathogenesis of metabolic cardiovascular diseases, mainly on the studies of T2D, atherosclerosis 
1. Assist the professor to carry out the daily administration and financial management of the laboratory.
2.Completed related biological experiments, such as cell stable strain construction, cell proliferation and apoptosis detection, AAV and lentivirus packaging, vector construction, WB, cell and tissue immunofluorescence, frozen section and HE staining, atherosclerosis modeling and evaluation, GTT/ITT, etc.
 \\
\end{SectionTable}

% --- Section: Talks and tutorials ---

\begin{SectionTable}{\headingfont Talks and tutorials}
Month Year &
Title of your most recent presentation \newline
\textit{Name of conference, workshop, seminar, etc., or a description} \\

Month Year &
Title of your second most recent presentation \newline
\textit{Name of conference, workshop, seminar, etc., or a description} \\

Month Year &
Title of your third most recent presentation \newline
\textit{Name of conference, workshop, seminar, etc., or a description} \\
\end{SectionTable}

% --- Section: Mentorship and service ---

\begin{SectionTable}{\headingfont Mentorship and service}
Month Year -- Present &
\textbf{Title of organization you are in (Name of your role)} \newline
Description of your responsibilities. Integer pretium semper justo. Proin risus. Nullam id quam. Nam neque. Phasellus at purus et lib ero lacinia dictum. \\

Month Year -- Month Year &
\textbf{Title of organization you were in (Name of your role)} \newline
Description of your responsibilities. Integer pretium semper justo. Proin risus. Nullam id quam. Nam neque. Phasellus at purus et lib ero lacinia dictum. \\
\end{SectionTable}

% --- Section: Professional society memberships ---

\begin{SectionTable}{\headingfont Professional memberships}
Year -- Present &
Name of professional society \newline
\textit{Short description or conferences you attended.} \\

Year -- Present &
Name of professional society \newline
\textit{Short description or conferences you attended.} \\
\end{SectionTable}

\begin{SectionTable}{\headingfont Technical skills}
& \textbf{Programming languages} \newline
Proficient in: language 1, language 2, language 3 \newline
Familiar with: language 4, language 5 \\

& \textbf{Software} \newline
\SPSS, snapgene,premier 5.0 \\

& \textbf{techniques} \newline
cell culture, western blot.
\end{SectionTable}

% --- Section: Other interests/hobbies ---

\begin{SectionTable}{\headingfont Other interests}
& Some of your hobbies, etc.
\end{SectionTable}

% --- End of CV! ---

\end{document}





