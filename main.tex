% --- Set document class and font size ---

\documentclass[letterpaper, 11pt]{article}

% --- Package imports ---

\usepackage{hyperref, enumitem, longtable, amsmath, array}
\usepackage{tabularx}

% --- Page layout settings ---

% Set page margins
\usepackage[left=0.7in, right=0.8in, bottom=.8in, top=0.8in, headsep=0in, footskip=.2in]{geometry}

% Set line spacing
\renewcommand{\baselinestretch}{1.2}

% --- Page formatting settings ---

% Indent for item space set to 0
\setlist[enumerate]{leftmargin=*}
\setlist[itemize]{leftmargin=*, after=\leavevmode\vspace*{-1\baselineskip}}

% Set link colors
\usepackage[dvipsnames]{xcolor}
\hypersetup{colorlinks=true, linkcolor=MidnightBlue, urlcolor=MidnightBlue}
\pagecolor{White}
\color{Black}

% Set font to Libertine, including math support
\usepackage{libertine}
\usepackage[libertine]{newtxmath}

% Remove page numbering
\pagenumbering{gobble}

% Define font size and color for section headings
\newcommand{\headingfont}{\Large\color{LimeGreen}}

% --- CV section settings ---

% Note: each section of this table (Education, Awards, Publications etc.) is 
% stored in a two-column table. The left-hand column is narrow (1 inch) and is 
% meant to store dates. The right-hand column is wide (5.2 inches) and stores 
% the main text.  Sections in which each entry might have multiple lines 
% (e.g., Education) are stored in a 'SectionTable' environment). Sections in 
% which each entry might just have one line are stored in a 'SectionTableSingleSpace'
% environment. The only difference between the two environments is the line 
% spacing between each entry. Both environments take one argument, which is the
% title of the section. See main document for how these environments are used.

% Define settings for left-hand column in which dates are printed
\newcolumntype{R}{>{\raggedleft}p{1in}}

% Define 'SectionTable' environment
\newenvironment{SectionTable}[1]{
	\renewcommand*{\arraystretch}{1.7}
	\setlength{\tabcolsep}{10pt}
	\begin{longtable}{Rp{5.2in}} & #1 \\}
{\end{longtable}\vspace{-.3cm}}

% Define 'SectionTableSingleSpace' environment
\newenvironment{SectionTableSingleSpace}[1]{
	\renewcommand*{\arraystretch}{1.2}
	\setlength{\tabcolsep}{10pt}
	\begin{longtable}{Rp{5.2in}} & #1 \\[0.6em]}
{\end{longtable}\vspace{-.3cm}}

% --- Document starts here ---

\begin{document}

% --- Name and contact information ---

\begin{SectionTable}{\Huge Ge Zhang} & 
gezhang24@outlook.com$\;\boldsymbol\;$ \newline
(0086)199-1654-4595
\end{SectionTable}

% --- Section: Research interests ---
\begin{SectionTable}{\headingfont Research Interests}
&\begin{itemize}[nosep, label={\textbullet}, parsep=1pt, noitemsep, topsep=0pt, before=\leavevmode\vspace*{-1\baselineskip}]
	\item Molecular mechanism of metabolic diseases including atherosclerosis, obesity, and diabetes.
	\item Tumor development and microenvironment.
	\item Molecular mechanism of platelet activation.
\end{itemize}
\end{SectionTable}

% --- Section: Education ---

\begin{SectionTable}{\headingfont Education}
2018 -- 2021 & 
\textbf{Fudan University} -- Shanghai, China \newline
MS in Biochemistry and Molecular Biology \newline 
Focus on the mechanism of platelet activation. \newline 
Mentor: Si Zhang. \textit{GPA: 3.2} \\

2014 -- 2018 & 
\textbf{Nantong University} -- Nantong, China \newline
BS in Biotechnology (Animals) \newline 
Focus on tumor genetics and epigenetics. \newline 
Mentor: Sheyu Lin. \textit{GPA: 3.6} \\

% --- Un-comment the next few lines if you want to include some courses you've taken ---

%& \textbf{Selected coursework}
%\begin{itemize}[itemsep=0pt, leftmargin=*]
%\item \textit{Statistics}: Asymptotic statistics, Mathematical statistics, Functional data analysis, High-dimensional statistics, Information theory
%\item \textit{Mathematics}: Measure theory, Functional analysis, Measure-theoretic probability with martingales
%\end{itemize}

\end{SectionTable}

% --- Section: Awards, scholarships, etc ---

\begin{SectionTableSingleSpace}{\headingfont Honors \& Scholarships}
2020 & 
Fudan University, Merit Scholarship \\

2020 &
Fudan University, Excellent Teaching Assistant Scholarship \\

2017 &
Nantong University, First-Class Scholarship of University \\

2016 &
Nantong University, Second-Class Scholarship of University  \\

2015 &
Nantong University, First-Class Scholarship of University 
\end{SectionTableSingleSpace}

% --- Section: Publications ---

\begin{SectionTable}{\headingfont Publications} 
2021 & 
\textbf{Activation of Platelet NLRP3 Inflammasome in Crohn's Disease} \newline
\textbf{Ge Zhang†}, He Chen†, Yifan Guo†, Wei Zhang, Qiuyu Jiang, Si Zhang, Liping Han, She Chen and Ruyi Xue*. \newline
\textit{Frontiers in Pharmacology. 2021 Jun 28;12:705325.} \href{https://pubmed.ncbi.nlm.nih.gov/34262463/}{link} \\

2021 & 
\textbf{The role of Sphingomyelin synthase 2 (SMS2) in platelet activation and its clinical significance} \newline
Y. Guo†, L Chang†, \textbf{G. Zhang†}, Z. Gao, H. Lin, Y. Zhang, L. Hu, S. Chen, B. Fan, S. Zhang, R. Xue. \newline
\textit{Thrombosis Journal. 2021 Apr 28;19(1):27.} \href{https://pubmed.ncbi.nlm.nih.gov/34262463/}{link} \\

2020 & 
\textbf{Methylation and serum response factor mediated in the regulation of gene ARRDC3 in breast cancer} \newline
S. Lin*, \textbf{G. Zhang*}, Y. Zhao, D. Shi, Q. Ye, Y. Li, S. Wang. \newline
\textit{Am J Transl Res. 2020 May 15;12(5):1913-1927.} \href{https://pubmed.ncbi.nlm.nih.gov/32509187/}{link}
\end{SectionTable}

% --- Section: Research experience ---

\begin{SectionTable}{\headingfont Research Experience}
2018 -- 2021 &
\textbf{The role of X in thrombopoiesis and platelet granule secretion (unpublished)} \newline
Mentor: Si Zhang (Fudan University)
\begin{enumerate}
	\item Investigated the mechanism of platelet count increase due to deficiency of X: Conducted analysis using various techniques including flow cytometry, transcriptome sequencing, and western blot to determine the effect of X deficiency on megakaryocyte proliferation, apoptosis and maturity. Results showed increased proliferation and advanced maturity of megakaryocytes, leading to elevated platelet count.
	\item The deficiency of X caused a decrease in platelet particles release: Through flow cytometry and ELISA analysis, it was determined that the deficiency of X resulted in a significant decrease in the release of platelet particles, including $\alpha$ granules, dense granules and lysosomes.
	\item Flotillin-2's interaction with X was determined: Interaction between X and flotillin-2 was confirmed through IP by anti-X antibody and mass spectrometry analysis, which was further verified through IP by anti-flotillin-2 antibody and immunofluorescence experiments. The interacting domains were identified through truncated body construction and protein structure simulation.
	\item Investigated the impact of X on chemotherapy: We investigated the correlation between platelet activation and X expression in clinical chemotherapy patients. Furthermore,  the effects of chemotherapy drugs on megakaryocyte proliferation, platelet count, and activation were explored in mouse models.
\end{enumerate} \\

2018 -- 2021 &
\textbf{The role of Sphingomyelin synthase 2 (SMS2) in platelet activation and its clinical significance} \newline
Mentor:  Si Zhang (Fudan University)
\begin{enumerate}
	\item Weakening of thrombus formation in vivo due to SMS2 deficiency: Our results demonstrated the significant role of SMS2 in thrombosis and hemostasis as evidenced by FeCl3-induced thrombus formation in mouse mesenteric and tail bleeding models.
	\item Adherence to informed consent protocols was maintained and blood samples were collected.
\end{enumerate} \\

2020 -- 2021 &
\textbf{Activation of Platelet NLRP3 Inflammasome in Crohn's Disease} \newline
Mentor: Si Zhang (Fudan University) 
\begin{enumerate}
	\item The manuscript was completed through a comprehensive analysis and interpretation of relevant data.
	\item Investigated the assembly of platelet NLRP3 inflammasome in patients with active Crohn's Disease (CD) and its correlation with platelet hyperactivity. Real-time PCR, western blot, flow cytometry, ELISA, co-IP and immunofluorescence experiments were conducted to uncover the results. Results suggest that the ROS-NLRP3 inflammasome-interleukin-1$\beta$-axis might contribute to platelet hyperactivity in active CD. 
\end{enumerate}  \\

2015 -- 2018 &
\textbf{Methylation and serum response factor mediated in the regulation of gene ARRDC3 in breast cancer} \newline
Mentor: Sheyu Lin (Nantong University)
\begin{itemize}
	\item Cell culture and viability, western blot, and animal experiments were successfully performed.
\end{itemize}
\end{SectionTable}

% --- Section: Teaching experience ---

\begin{SectionTable}{\headingfont Teaching Experience}
Fall 2019 & 
\textbf{Biochemical experiment course, Teaching assistant(Fudan University)} \\

Summer  2020 & 
\textbf{Experimental course in molecular genetics, Teaching assistant(Fudan University)}
\begin{itemize}
	\item Assisted in the preparation of lessons and pre-experiments under the guidance of laboratory instructors. Provided guidance and supervision during laboratory experiments, and evaluated students' laboratory reports. 
\end{itemize}
\end{SectionTable}

% --- Section: work experience ---

\begin{SectionTable}{\headingfont Work Experience}

2022.09-present &
\textbf{Abcam (Mission Support)} -- Shanghai, China
\begin{itemize}[itemsep=4pt, label={\textbullet}, parsep=1pt]
	\item Handled customer inquiries through phone calls and responded to product information inquiries, recommended appropriate products, and offered optimization suggestions to resolve customer experiment issues
	\item Managed customer complaint correspondence, offered solutions for customer experiments, maintained communication with customers, and effectively resolved customer concerns
	\item Proficient in fundamental molecular biology and cell biology laboratory techniques, capable of offering effective experimental design and resolution for customers
	\item Demonstrated strong team collaboration through effective communication to efficiently resolve customer inquiries and issues.
\end{itemize} \\

2021.07-2022.07 &
\textbf{Fudan University  (Research Assistant)} -- Shanghai, China \newline
Focus on the pathogenesis of metabolic cardiovascular diseases, particularly in the studies of type 2 diabetes and atherosclerosis. 
\begin{itemize}[itemsep=4pt, label={\textbullet}, parsep=1pt]
	\item Supported the head of the laboratory in overseeing daily administrative and financial operations
	\item Conducted a variety of biology experiments, including cell line establishment, analysis of cell proliferation and apoptosis, packaging of AAV and lentivirus, vector construction, Western blot, immunofluorescence analysis of cells and tissues, cryosectioning and HE staining, modeling and evaluation of atherosclerosis, and performing glucose tolerance and insulin tolerance tests.
\end{itemize}
\end{SectionTable}

\begin{SectionTable}{\headingfont Technical Skills}

& \textbf{Software} \newline
SnapGene, Premier 5.0, FlowJo, Graphpad Prism, Photoshop, Adobe Illustrator.\\

& \textbf{Techniques} \newline
Western blot, Cell culture, Cell proliferation, Cell apoptosis, Cell cycle, Cell invasion, Transwell, Lentivirus/AAV packaging, qPCR, IHC, ICC, Flow cytometry, Magnetic bead sorting, Co-IP, EMSA, ChIP, ELISA, GST-pull down, Tail infection, Peritoneal injection, Canthus venous plexus injection, Cryosectioning, Plasmid construction, Platelet separation, Platelet aggregation, Clot retraction, Ferric chloride-induced mesenteric arterial injury model, Mouse tail bleeding, Expression and purification of prokaryotic proteins, Transcriptome sequencing. \\
\end{SectionTable}

% --- End of CV! ---

\end{document}





